\documentclass{article}

\usepackage{minted}

\setlength{\parindent}{0pt}

\begin{document}

\title{NameSayer User Manual}
\author{Jordan Sim-Smith \\
		The University of Auckland \\
		\and 
		Joshua Fu \\
		The University of Auckland \\
		}

\maketitle

\clearpage

\tableofcontents

\clearpage

\section{Setup}

This page describes the system requirements to run NameSayer, and will guide the user through system setup and the launch of the 
NameSayer application.

\subsection{Required Software}

NameSayer has been designed to run on a Linux based Operating System, relying on bash system calls. To run NameSayer, please
ensure that you are running a Linux distribution with a Graphical User Interface. \\

NameSayer also requires a \textbf{Java Runtime Environment} to run in and the \textbf{ffmpeg} program used for audio manipulation. \\

Using a terminal, Java 8 can be installed as follows:
\begin{minted}{bash}
$ sudo add-apt-repository ppa:webupd8team/java
$ sudo apt update
$ sudo apt install oracle-java8-installer
\end{minted}

Similarly, ffmpeg can be installed as follows:
\begin{minted}{bash}
$ sudo apt update
$ sudo apt install ffmpeg
\end{minted}

\subsection{Required Hardware}
The NameSayer application involves the recording and playback of audio. Therefore, NameSayer requires that the user's
computer has either an internal or external microphone that is turned on. The user should ensure that their computer
has audio playback capability (speakers) and that they are turned on.

\subsection{Launching NameSayer}
Before launching NameSayer, the names database must be placed in a folder called \textbf{names} in the same directory as 
the executable jar. \\

NameSayer must be run from the command line. The user must first navigate into the NameSayer folder. Then the application
can be launched as follows: 

\begin{minted}{bash}
$ java -jar namesayer.jar
\end{minted}

\section{Home Screen}

\section{View the Database}

\section{Search the Database}

\section{Upload a Custom List}

\section{Name Practice}

\section{View Past Attempts}

\section{Special Features}
To enhance the user's experience with NameSayer, several special features have been added. These features have been designed
with both the target user and all other users in mind.

\subsection{Streaks}
A NameSayer streak occurs when the user has been using the program for at least 2 days consecutively. Every consecutive day 
that a user uses NameSayer, the streak will be incremented. If a user does not log onto NameSayer for a whole day, the 
streak will be reset. Streaks are an effective form of motivation for the user to keep returning to the application, 
improving their name pronunciation. \\

Upon opening the NameSayer application, if a streak is present, the user will be congratulated on their consistent efforts.
A pop up window will appear, with an affirmation as well as the length of the streak (number of days).

\subsection{Selective Rating}
NameSayer offers the option to either practice a name individually or amongst a group of other names, for example,
practising a persons's first and last name together. Upon listening to a name in the database of low quality, the user may
choose to flag this name to the database maintainer using the \textbf{Bad Recording} button. \\

When practising more than one name
at once, the user may require that one or more of the played names be flagged as low quality. The \textbf{Bad Recording} button
prompts the user to select the name that they would like to flag, instead of flagging the whole selection or none of it.

\subsection{Multiple File Upload}
One of the features of NameSayer is that a user can upload a text file containing a list of names to be practised together or 
sequentially. NameSayer extends this functionality such that multiple lists can be combined together, instead of restricting
the user to a single file upload.

\subsection{Live Search Validation}
Another feature of NameSayer is an explicit search for a name or group of names. To enhance usability, NameSayer filters 
the user's search query in real time for invalid characters. Thus, the user's search will not be invalidated because of disallowed
characters. A live filter is also much more intuitive for the user. Instead of typing out an entire query to be told that it is invalid,
a live filter lets a user know right away what can be searched for and what can't be.

\end{document}